\section*{Аннотация}

Измерена зависимость мощности нагревателя от разности температур воздуха в диапазоне от 2 до $\SI{10}{\celsius}$ при двух расходах воздуха при атмосферном давлении. (не важная информация для работы, нужно писать про измерение теплоемкости) Измерения проводились с помощью калориметра с встроенным нагревателем, через который прокачивался воздух. В результате обнаружено, что  мощность нагревателя пропорциональна разности температур воздуха. По угловому коэффициенту полученной зависимости вычислена удельная теплоёмкость воздуха при постоянном давлении $(1063 \pm 13)  \frac{\text{Дж}}{\text{кг}\cdot\text{K}}$. 