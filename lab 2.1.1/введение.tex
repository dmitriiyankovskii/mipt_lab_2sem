\section{Введение}

В современных условиях особый интерес представляет повышение энергоэффективности зданий и сооружений, оптимизация промышленных процессов, поэтому точный расчет тепловой нагрузки приобретает первостепенное значение. Тепловая нагрузка представляет собой количество энергии, которое необходимо подвести/отвести для поддержания заданных температурных условий внутри помещения, технологического аппарата. Тепловая нагрузка является ключевым параметром при проектировании систем отопления и вентиляции. Одним из фундаментальных параметров, используемых при расчете тепловой нагрузки, является удельная теплоемкость $c_p$ воздуха при постоянном давлении. Целью настоящей работы было измерение удельной теплоемкости воздуха при атмосферном давлении. (добавить конкретики, описать более подробно решаемую задачу с помощью полученной теплоемкостью)