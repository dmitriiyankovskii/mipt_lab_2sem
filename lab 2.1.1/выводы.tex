\section{Выводы}

По графикам зависимости $N(\Delta T)$ определено, что при небольшом нагреве $\Delta T \ll T$ мощность тепловых потерь пропорциональна разности температур $\Delta T$.

Определено, что при увеличении расхода воздуха увеличивается угловой коэффициент наклона графика $N(\Delta T)$.

Определено, что доля тепловых потерь увеличивается при уменьшении расхода воздуха.

Вычислено значение удельной теплоемкости воздуха при постоянном давлении $c_p = (1063 \pm 13)\frac{\text{Дж}}{\text{кг} \cdot \text{К}}$, которое оказалось завышенным по сравнению с табличным значением. Это связано с погрешностью измерения разности температур $\Delta T$ с помощью цифрового вольтметра. Значения считывались в момент, когда система еще не достигла стационарного состояния, в котором показания вольтметра не изменяются.

