\section{Методика}

Известно из литературы (\href{https://old.mipt.ru/education/chair/physics/S_II/lab/211.pdf}{методическое пособие}), что удельная теплоемкость воздуха, прокачиваемого через калориметр, зависит от мощности нагревателя по следующему закону
\begin{equation}
    c_p = \frac{N - N_\text{пот}}{q\Delta T} \label{eq: cp}
\end{equation}
где $N$ - мощность нагревателя, $N_\text{пот}$ - мощность тепловых потерь, $q$ - массовый расход воздуха, $\Delta T$ - разность температур воздуха на входе и выходе из калориметра.

При малых расходах воздуха и достаточно большом диаметре трубы калориметра перепад давления на ее концах мал, поэтому $P_1 \approx P_2 = P_0$. Из этого следует, что $c_p$ в выражении \eqref{eq: cp} является удельной теплоемкостью воздуха при постоянном давлении.

При $\Delta T \ll T$ можно считать, что мощность тепловых потерь пропорциональна разности температур воздуха $N_\text{пот} = \alpha \Delta T$. При этом выражение \eqref{eq: cp} принимает вид
\begin{equation}
    N = (c_p q + \alpha)\Delta T \label{eq: N(delta T)}
\end{equation}
$\alpha$ - некоторая константа.


Для определения удельной теплоемкости $c_p$ воздуха необходимо измерить зависимости мощности $N$ нагревателя от разности температур $\Delta T$ воздуха на концах трубы калориметра при двух расходах $q$ воздуха. Угловые коэффициенты полученных зависимостей $k = c_p q + \alpha$ являются линейными функциями от удельной теплоемкости $c_p$. Значит $c_p$ определяется как угловой коэффициент графика зависимости $k$ от расхода $q$ воздуха. 