\section{Результаты и их обсуждение}
Экспериментальная установка состоит из форвакуумного и диффузионного насосов, к которым присоединена система из трубок и сосудов. Подробное описание установки см. в \nameref{Приложение 3}.

Для измерения давления в разных частях системы использовались термопарный, масляный и ионизационный манометры. Описание их устройства см. в \nameref{Приложение 3}.

Перед определением скорости откачки были измерены объемы форвакуумной  $V_\text{фв}$ и высоковакуумной $V_\text{вв}$ частей установки (см. \nameref{Приложение 4}). Они составили $V_\text{фв} = (2010\pm 40)\text{см}^3, V_\text{вв} = (1150\pm 30)\text{см}^3$.

Для определения скорости откачки высоковакуумной части вся установка была откачана форвакуумным насосом до давления $3\cdot10^{-2} \text{торр}$, и затем высоковакуумная часть установки была отделена от форвакуумной путем закрытия крана $K_6$ и откачана диффузионным насосом до давления $P_\text{пр} = 5.6\cdot 10^{-5} \text{торр}$, измеренного с помощью ионизационного манометра.

Затем высоковакуумная часть была отсоединена от насоса на некоторое время для ухудшения вакуума и присоединена обратно. Была измерена зависимость давления $P$ воздуха в части от времени $t$ откачки. Результаты измерений приведены в таблице \eqref{tab:t1}. По результатам измерений построен график зависимости    $\ln({P-P_\text{пр}})(t)$ \eqref{fig:lnP(t)}.
\begin{figure}[ht]
\center\begin{gnuplot}[terminal = pdf]
    set xl "t, c"
    set yl "ln (P - P_{пр})"
    set xr [0:38]
    set format y "%.1f"
    set grid
    f(x) = k*x + b
    fit f(x) "txt/P(t)improve" u ($1 - 48):(log($2 - 0.64)) via k,b
    plot "txt/P(t)improve" u ($1 - 48):(log($2 - 0.64)):(0.5):(0.05 / (2)) w xye pt 0 lc "red" lw 2 t "Данные", \ 
    f(x) lw 3 pt 7 ps 0.5 t "Аппроксимация"
\end{gnuplot} 
\vspace{-20pt}
\caption{Зависимость разности давления $P$ в системе и предельного давления $P_\text{пр}$ измеренная в мм. рт. ст. от времени $t$ в логарифмическом масштабе при улучшении вакуума.}
\label{fig:lnP(t)}
\end{figure}

С помощью метода наименьших квадратов определён угловой коэффициент $k = (-0.253 \pm 0.010)\text{c}^{-1}$ графика зависимости $\ln({P-P_\text{пр}})(t)$, и, зная объём $V_\text{вв}$ откачиваемой системы, из \eqref{eq: ln(P)(t)} определена скорость откачки $W = (3.17\pm0.13)\cdot10^2\frac{\text{см}^3}{c}$.
Таким образом, данный метод позволяет измерить скорость откачки с точностью $4.1\%$.

Также полезно знать величину потока $Q_\text{н}$ газа, поступающего из диффузионного насоса в откачиваемую систему. Для этого был закрыт кран $K_3$, чтобы прекратить откачку высоковакуумной части установки. При помощи ионизационного манометра была измерена зависимость давления $P$ от времени $t$ при ухудшении вакуума. Результаты измерений приведены в таблице 2 (\nameref{Приложение 6}). График зависимости $P(t)$ представлен на рис. \eqref{fig: P(t)}.

\begin{figure}[ht]
\center\begin{gnuplot}[terminal = pdf]
    set xl "t, c"
    set yl "P, мм. рт. ст. · 10^{−4}"
    set format y "%.2f"
    set grid
    set key b
    f(x) = k*x + b
    fit f(x) "txt/P(t)worse" u 1:2 via k,b
    plot "txt/P(t)worse" u 1:2:(0.5):(0.05) w xye pt 0 lc "red" lw 2 t "Данные", \ 
    f(x) lw 3 pt 7 ps 0.5 t "Аппроксимация"
\end{gnuplot}
\vspace{-20pt}
\caption{Зависимость давления $P$ в системе от времени $t$ при ухудшении вакуума.}
\label{fig: P(t)}
\end{figure}

График аппроксимируется линейной зависимостью, из этого следует, что для описания зависимости давления газа от времени при ухудшении вакуума справедливо выражение \eqref{eq: dP_dt}.

С помощью метода наименьших квадратов определён угловой коэффициент $k = (6.4 \pm 1.0) \cdot 10^{-6}~\text{мм. рт. ст.}/ \text{с}$. Из \eqref{eq: Q_n} определён массовый поток $Q_\text{н}$ газа, который составил

\[Q_\text{н} = (1.23 \pm 0.15) \cdot 10^{-2}~\text{см}^3/\text{с}.\]


Для определения скорости $W_\text{теч}$ откачки системы диффузионным насосом при наличии течи, был открыт кран $K_6$, чтобы появилась искусственная течь через капилляр. Было измерено установившееся давление $P_\text{уст} = (5.9\pm0.2)\cdot 10^{-5} \text{торр}$ в высоковакуумной части и давление $P_\text{фв}$ со стороны форвакуумного насоса.
По измеренным давлениям, а также параметрам установки, таким как температура воздуха $T = 298.7K$, длина $l = (10.8\pm0.1)\text{см}$, радиус $r=(0.40\pm0.05)\text{мм}$ капилляра был определен массовый расход воздуха через капилляр и по \eqref{eq: W_tech} определена скорость откачки. 
$W_\text{теч} = (27\pm10)\frac{\text{см}^3}{\text{с}}$.

Найденное значение скорости откачки оказалось меньше при наличии течи, чем при её отсутствии. Это можно обосновать тем, что после создания течи эффективная пропускная способность системы вместе с капилляром снизилась, что не позволило достичь максимальной производительности насоса. Из этого следует, что скорость откачки зависит не только от производительности насоса, но и от пропускной способности трубопровода.