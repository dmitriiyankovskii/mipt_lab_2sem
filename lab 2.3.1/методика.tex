\section{Методика}

Из литературы \cite{Sivuhin} известно, что давление $P$ в сосуде при откачке зависит от времени $t$ так
\begin{equation}
    P = P_0\exp({-\frac{W}{V}t}) + P_\text{пр} \label{eq: P(t)}
\end{equation}
где $P_0$ - начальное давление в системе, $P_\text{пр}$ - предельное давление воздуха при откачке, $V$ - объем откачиваемой системы. Подробный вывод см. в \nameref{Приложение 1}.

Если прологарифмировать правую и левую части выражения \eqref{eq: P(t)}, получится
\begin{equation}
    \ln({P - P_\text{пр}}) = \ln{P_0} -\frac{W}{V}t \label{eq: ln(P)(t)}
\end{equation}

Для количества газа $\frac{d(PV)}{dt}$, протекающего по трубке при высоком вакууме в единицу времени, справедливо выражение

\begin{equation}
    \frac{d(PV)}{dt} = \frac{4}{3}r^3\sqrt{\frac{2\pi RT}{\mu}}\frac{P_2-P_1}{l} \label{eq: C}
\end{equation}
где $r$ - радиус трубы, $T$ - температура газа, $R$ - универсальная газовая постоянная, $\mu$ - молярная масса газа, $l$ - длина трубы, $P_2-P_1$ - разность давления на концах трубы.

Чтобы определить скорость откачки $W$ системы, необходимо измерить зависимость давления $P$ воздуха в системе от времени $t$ при откачке. Скорость откачки определяется через угловой коэффициент графика зависимости $\ln({P - P_\text{пр}})(t)$ согласно выражению \eqref{eq: ln(P)(t)}.

Чтобы определить поток $Q_\text{н}$ газа, текущего из насоса в откачиваемый объем, необходимо измерить зависимость давления $P$ газа в установке от времени $t$ при ухудшении вакуума. Для данного процесса выражение \eqref{eq: 1} принимает вид 
\begin{equation}
    V_\text{вв}dP = (Q_\text{н} + Q_\text{д})dt \label{eq: dP_dt}
\end{equation}
Пренебрегая явлением десорбции газа через стенки трубы, поток $Q_\text{н}$ можно вычислить через угловой коэффициент $k$ графика зависимости $P(t)$ согласно следующему выражению
\begin{equation}
    Q_\text{н} = V_\text{вв}k \label{eq: Q_n}
\end{equation}

Чтобы определить скорость $W_\text{теч}$ откачки воздуха при наличии искусственной течи через капилляр, необходимо измерить установившееся давление на концах капилляра и массовый расход $\frac{d(PV)_\text{кап}}{dt}$ воздуха через него с помощью выражения \eqref{eq: C}. Искомая скорость откачки находится из \eqref{eq: W_tech}.

\begin{equation}
    W_\text{теч} = \frac{\frac{d(PV)_\text{кап}}{dt}}{P_\text{уст} - P_\text{пр}} \label{eq: W_tech}
\end{equation}

Подробный вывод выражения \eqref{eq: W_tech} см. в \nameref{Приложение 2}.

