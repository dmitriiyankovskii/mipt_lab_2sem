\section{Выводы}
Определены значения вязкости воздуха при комнатной температуре и атмосферном давлении при течении через тонкие трубки. $\eta = (2.0\pm 0.2)\text{Па}\cdot\text{c}$ для трубки диаметром $d = 3.95\text{мм}$ и $\eta = (1.7\pm 0.4)\text{Па}\cdot\text{c}$ для трубки диаметром $d = 5.05\text{мм}$. Полученные значения в пределах погрешности совпали с табличным значением $\eta = 1.83\text{Па}\cdot\text{c}$. Таким образом, формула Пуазейля позволяет оценить вязкость воздуха в тонкой трубке с точностью ....

Значения длин установления потока, измеренные по графику $P(x)$ для обеих трубок ($l_\text{уст1} = 40\text{см}, l_\text{уст2} = 50\text{см}$), разошлись со значениями, определенными из теоретических предположений ($l_\text{уст1теор} = 39.5\text{см}, l_\text{уст2теор} = 50.5\text{см}$), менее чем на 2\%. Это показывает, что данный метод оценки длины установления позволяет определить $l_\text{уст}$ с точностью до 2\%.

Полученные значения коэффициентов $\beta_\text{л} = 1.7, \beta_\text{т} = 3.0$ не сошлись с теоретическими значениями  $\beta_\text{л.теор} = 4, \beta_\text{т.теор} = 2.5$. Из этого можно сделать вывод, что теоретическая модель описания турбулентного течения, основанная на том, что флуктуации скорости по порядку величины совпадают со средней скоростью потока и элементы жидкости равномерно перемешиваются по сечению трубы, и метод размерностей для ламинарного течения неприменимы (в рамках использованных приближений ) для описания зависимости расхода газа от радиуса трубки в условиях данного эксперимента.