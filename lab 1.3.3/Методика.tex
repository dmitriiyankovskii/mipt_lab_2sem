\section{Методика}
Из литературы \cite{Sivuhin} известно, что зависимость  расхода $Q$ газа, ламинарно текущего по трубе радиуса $R$, от перепада $\Delta P$ давления на концах трубы выражается через формулу Пуазейля следующим образом
\begin{equation}
    Q = \frac{\pi R^4 \Delta P}{8\eta l} \label{eq: Q}
\end{equation}
где $\eta$ - коэффициент вязкости газа.

Давление в трубе является линейно убывающей функцией координаты
\begin{equation}
    P(x) = P_0 - \frac{\Delta P}{l}x \label{eq: P(x)}
\end{equation}
где $P$ - давление в трубе на расстоянии x от ее начала, $P_0$ - давление газа на входе в трубу, $\frac{\Delta P}{l}$ - перепад давления на длине трубы $l$.

Характер течения (ламинарное, турбулентное) определяется безразмерным параметром - числом Рейнольдса
\begin{equation}
    Re = \frac{\rho u a}{\eta} \label{eq: Re}
\end{equation}
где $\rho$ - плотность среды, $u$ - характерная скорость потока, $a$ - характерный размер системы (размер, на котором существенно меняется скорость течения), $\eta$ - коэффициент вязкости среды.

Это число имеет смысл отношения кинетической энергии $K$ движения элемента объема жидкости к потерям энергии $A_\text{тр}$ из-за трения в нем
\begin{equation}
    Re \propto \frac{K}{A_\text{тр}}
\end{equation}
При достаточно малых значениях $Re$ в потоке доминируют вязкие силы трения, и течение является ламинарным. С ростом $Re$ может быть достигнуто критическое значение $Re_\text{кр}$, при котором характер течения сменяется с ламинарного на турбулентный.

В целях упрощения теоретической модели воздух считается несжимаемым, то есть его плотность $\rho = const$. Такое приближение допустимо, так как в условиях эксперимента перепад давления на концах трубы мал по сравнению с внешним атмосферным давлением ($\Delta P \ll P$).

Если на вход трубы поступает течение, распределение скоростей которого не является пуазейлевским, то профиль течения установится через некоторое расстояние $l_\text{уст}$ за счет сил вязкого трения
\begin{equation}
    l_{\text{уст}} \approx 0.2R \cdot Re \label{eq: l_ust}
\end{equation}
$R$ - радиус трубы.

Расход газа при турбулентном течении выражается через параметры системы следующим образом
\begin{equation}
    Q = const \cdot R^{\frac{5}{2}} \sqrt{\frac{\Delta P}{\rho l}} \label{eq: Q1}
\end{equation}
$Q$ - расход воздуха, $R$ - радиус трубы, $\frac{\Delta P}{l}$ - перепад давления воздуха на длине трубы $l$, $\rho$ - плотность воздуха.

Для определения вязкости $\eta$ воздуха при его ламинарном протекании через трубу необходимо измерить зависимость расхода $Q$ воздуха от перепада давления $\Delta P$ на выделенном участке трубы длиной $l$. Чтобы убедиться, что $\eta$ зависит только от вещества, но не от параметров системы необходимо провести измерения на трубках разного диаметра.

Затем необходимо измерить распределение давления вдоль трубки $P(x)$ при фиксированном расходе $Q$ воздуха. Измерения проводятся на нескольких трубках. По графикам зависимости $P(x)$ можно определить длину установления $l_\text{уст}$ на каждой трубке. Течение считается установившимся на участке графика, где зависимость приняла линейный вид. (наоборот)

Чтобы проверить выполнимость выражений \eqref{eq: Q} и \eqref{eq: Q1} для описания ламинарного и турбулентного течений, необходимо измерить зависимость расхода $Q$ воздуха от радиуса $R$ трубы при заданном градиенте давления $\frac{\Delta P}{l}$. Необходимо подобрать (зачем) некоторое значение градиента $\frac{\Delta P}{l}$, при котором обеспечивается ламинарность на всех трубках. На каждой трубке подберём значение расхода $Q$, при котором градиент $\frac{\Delta P}{l}$ соответствует выбранному. Аналогично выберем значение градиента $\frac{\Delta P}{l}$, при котором в трубках наблюдается турбулентное течение, и проведём измерение $Q(R)$. 

Построив график зависимости $Q(R)$ в двойном логарифмическом масштабе $\ln{Q}(\ln{R})$, определим степень $\beta$ в зависимости $Q \propto R^\beta$ как угловой коэффициент.