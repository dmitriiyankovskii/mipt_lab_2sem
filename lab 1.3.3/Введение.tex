\section{Введение}

В современной автомобильной промышленности аэродинамика кузова играет определяющую роль в снижении расхода топлива и уменьшении выбросов вредных веществ. Традиционно акцент делался на оптимизации формы кузова, использовании спойлеров и закрытии днища. Учет влияния вязкости воздуха (необходимо) новые возможности для повышения эффективности и достижения более высоких результатов.

Коэффициент вязкости воздуха является физическим свойством, определяющим силы трения, возникающие при движении тела в воздушной среде. В аэродинамике кузова этот параметр влияет на коэффициент аэродинамического сопротивления и общую эффективность обтекания кузова воздушным потоком.

Традиционные методы расчета аэродинамических характеристик часто основываются на упрощенных моделях (ссылка), пренебрегающих влиянием вязкости воздуха. С развитием вычислительной гидродинамики и увеличением вычислительной мощности появилась возможность детального моделирования воздушных потоков и учета влияния вязкости воздуха.
(Чтобы осуществить моделирование необходимо знать коэфф вязкости)
Целью настоящей работы являлось вычисление коэффициента вязкости воздуха при атмосферном давлении.