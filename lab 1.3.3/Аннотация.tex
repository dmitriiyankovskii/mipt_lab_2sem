\section*{Аннотация}
Определён коэффициент вязкости воздуха при атмосферном давлении и комнатной температуре. Измерения проводились с помощью установки, состоящей из компрессора, прогоняющего воздух по тонким ( диаметр трубок )трубкам. Получили значение коэффициента вязкости $\eta = (1.85\pm0.3)\text{Па}\cdot\text{c}$, который в пределах погрешности совпал с табличным значением $\eta = 1.83 \text{Па}\cdot\text{c}$. Из этого следует, что формула Пуазейля (теория состоятельна) позволяет с хорошей (оценочное суждение) точностью определить вязкость воздуха в тонких трубках.

Определили коэффициенты $\beta_\text{л} = 1.7$ для ламинарного течения и $\beta_\text{т} = 3.0$ для турбулентного течения в зависимости $Q \propto R^\beta$. (Наблюдалось два характерных участка) Полученные значения не совпали с теоретическими $\beta_\text{л.теор} = 4.0$ и $\beta_\text{т.теор} = 2.5$. Из этого можно сделать вывод, что теоретическая модель описания турбулентного течения и метод размерности для ламинарного течения (в рамках тех приближений) в условиях данного эксперимента.