\section*{Аннотация}

 Измерены коэффициенты взаимной диффузии гелия и воздуха при комнатной температуре $\SI{22}{\celsius}$ при давлениях смеси от 40 до 300 торр. Измерения (разности концентраций проводились по изменению теплопроводности смеси) проводились с помощью системы из двух соединенных сосудов, в которые накачивалась смесь газов. В результате обнаружено, что коэффициент диффузии обратно пропорционален давлению. Методом экстраполяции полученной зависимости коэффициента диффузии от давления было получено, что при атмосферном давлении он составляет $(0.49 \pm 0.03)\frac{{\text{см}}^2}{\text{c}}$. Это означает, что при возникновении утечки $\alpha$ активных изотопов у жителей населенного пункта, удаленного на расстоянии 1 км. имеется ... времени на эвакуацию. 