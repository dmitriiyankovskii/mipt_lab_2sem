\section{Методика}

Известно из литературы (\href{https://old.mipt.ru/education/chair/physics/S_II/lab/rabota221/221.pdf}{методическое пособие}), что разность ($\Delta n$) концентраций  примеси гелия в двух резервуарах, между которыми наблюдается диффузия, изменяется по экспоненциальному закону
\begin{equation}
     \Delta n = \Delta n_0 e^{-\frac{t}{\tau}} \label{eq:delta_n_t}
\end{equation}
где $\tau$ - характерное время релаксации, 
$\tau = \frac{VL}{2DS}$, $V$ - объем резервуаров со смесью газов, $L$ - длина трубки, в которой наблюдается диффузия, $S$ - площадь сечения трубки, $D$ - коэффициент диффузии.

Для определения коэффициента $D$ диффузии  необходимо измерить зависимость разности $\Delta n$ концентраций  от времени $t$ диффузии, чтобы определить $\tau$. 

Основная задача состоит в том, чтобы определить $\Delta n$ гелия в сосудах в каждый момент времени. В работе использована зависимость теплопроводности смеси газов $k$ от ее состава. В общем случае теплопроводность сложным образом зависит от концентрации, однако при малой разности концентраций в сосудах ($\Delta n$) можно ожидать, что разность теплопроводностей ($\Delta  k$) будет прямо пропорциональна $\Delta n$.
\begin{equation}
     \Delta k = k(n_2)-k(n_1) \approx const \cdot \Delta n \label{eq: Delta_k}
\end{equation}
Эксперименты показывают, что если доля примеси гелия составляет менее 15\%, отклонение от линейной зависимости не превышает 0.5\%, что для наших целей вполне достаточно. Более подробное описание можно найти в \href{https://old.mipt.ru/education/chair/physics/S_II/lab/rabota221/221.pdf}{методическое пособие}.


В эксперименте использованы два нагревательных элемента - тонкие платиновые проволочки, расположенные в обоих сосудах. По проволочкам пропускается электрический ток. Тепло от проволочки к стенкам сосуда передается в основном за счет теплопроводности газа. При заданной мощности нагревания приращение температуры проволочки, и следовательно, приращение ее сопротивления  пропорциональны теплопроводности газа.
\begin{equation}
     R(t)=R_{273}\cdot(1+\alpha T) \label{eq: R_t}
\end{equation}
где $\alpha$ - температурный коэффициент сопротивления материала,  $\alpha = \frac{1}{R_{273}}\frac{dR}{dT}$.

Так как $U = IR$ - напряжение на проволоке, и оно пропорционально сопротивлению, то оно также пропорционально теплопроводности газа.
Это означает, что разность напряжений на проволоках пропорциональна разности теплопроводностей газов в сосудах и пропорциональна разности концентраций примеси в сосудах (выражение \eqref{eq: Delta_k}).
\begin{equation}
     U \propto \Delta k \propto \Delta n \label{eq: prop}
\end{equation}
Из \eqref{eq:delta_n_t} и \eqref{eq: prop} следует, что 
\begin{equation}
     U = U_0 e^{-\frac{t}{\tau}} \label{eq: U_t}
\end{equation}
где $U_0$ - разность напряжений на проволоках в начале измерений.
Окончательно имеем, что для определения коэффициента $\tau$ в выражении \eqref{eq:delta_n_t} необходимо измерить зависимость разности напряжений на нагревательных элементах от времени, удовлетворяющую закону \eqref{eq: U_t}.
Непосредственно коэффициент $\tau$ определяется как угловой коэффициент графика $ln(U)(t)$ 
\begin{equation}
     ln(U) = ln(U_0) - \frac{t}{\tau} \label{eq: ln(U)_t}
\end{equation}
Значение коэффициента $D_{\text{атм}}$ диффузии при атмосферном давлении может быть получено путем экстраполяции зависимости $D(\frac{1}{P})$ в сторону атмосферного давления. 