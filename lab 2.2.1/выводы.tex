\section{Выводы}

     По графику зависимости $ln(U)(t)$ определено, что с ростом давления смеси гелия и воздуха коэффициент диффузии уменьшается. 

     
     По графику $D(\frac{1}{P})$ определен характер зависимости. Коэффициент диффузии - линейная функция от $\frac{1}{P}$. Предположение теории оказалось верно.

     
     Определен коэффициент диффузии гелия и воздуха при атмосферном давлении $D_{\text{атм}} = (0.49 \pm 0.03)\frac{\text{см}^2}{c}$

