\section{Введение}

С 1954 года атомные электростанции активно используются в промышленности. Известны случаи аварий на станциях, сопровождающиеся выбросом радиоактивных изотопов, в результате распада которых образуются $\alpha$-частицы (ядра атома гелия). В случае радиоактивных выбросов фон может распространиться на большие расстояния, делая непригодным для жизни загрязненные области до полного распада радиоактивных компонентов. В этом случае необходимо иметь возможность оценить время, необходимое для эвакуации людей из ближайших к выбросу территорий. Проводить натурный эксперимент опасно, поскольку территории, выделенные под эксперимент в течение длительного времени будут загрязнены и могут подвергнуть риску флору и фауну. Целесообразно провести лабораторный эксперимент, в котором $\alpha$-частицы будут заменены гелием, и оценить времена диффузии гелия в воздухе. Поскольку детектировать изменение концентрации гелия можно лишь в случае, когда она сопоставима с концентрацией воздуха, имеет смысл проводить такой эксперимент при пониженном давлении воздуха, чтобы обеспечить должную чувствительность установки, а затем экстраполировать в область атмосферного давления. Целью настоящей работы было измерение коэффициента диффузии гелия в воздухе при пониженном давлении. 
