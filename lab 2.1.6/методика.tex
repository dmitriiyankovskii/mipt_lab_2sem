\section{Методика}

Известно из литературы \cite{Sivuhin}, что коэффициент Джоуля-Томсона для дифференциального эффекта ($\Delta T \ll T$, $\Delta P \ll P$) очень разреженного газа зависит от параметров исследуемого газа по следующему закону
\begin{equation}
    \mu_\text{д-т} = \frac{\Delta T}{\Delta P} \approx \frac{\frac{2a}{RT}-b}{c_p} \label{eq: mu}
\end{equation}
где $\Delta T$ - разность температур газа с двух сторон от пористой перегородки, $\Delta P$ - разность давлений газа после прохождения пористой перегородки и до прохождения, $a, b$ - коэффициенты уравнения Ван-дер-Ваальса газа, $R$ - универсальная газовая постоянная, $R = 8.31\frac{\text{Дж}}{\text{моль}\cdot\text{К}}$, $c_p$ - молярная теплоемкость газа при постоянном давлении.  

Температура инверсии, при которой эффект Джоуля-Томсона меняет знак, вычисляется из выражения \eqref{eq: T_inv}.
\begin{equation}
    T_\text{инв} = \frac{2a}{Rb} \label{eq: T_inv}
\end{equation}

Для определения коэффициента $\mu_\text{д-т}$ Джоуля-Томсона необходимо измерить зависимость разности $\Delta T$ температур, на которую изменяется температура газа при протекании через пористую перегородку, от разности $\Delta P$ давлений на концах трубки. $\mu_\text{д-т}$ вычисляется как угловой коэффициент полученного графика $\Delta T (\Delta P)$. Зависимость $\Delta T (\Delta P)$ необходимо измерить для нескольких начальных значений температуры $T$ газа, чтобы по графику $\mu_\text{д-т}(\frac{1}{T})$ определить значение $a$ через угловой коэффициент $\frac{2a}{Rc_p}$ и $b$ через свободный член $\frac{-b}{c_p}$. Затем температуру инверсии можно найти из выражения \eqref{eq: T_inv}.