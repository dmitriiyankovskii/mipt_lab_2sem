\section*{Аннотация}
Определены коэффициенты в уравнении Ван-дер-Ваальса для углекислого газа, и определена температура инверсии углекислого газа. 
Измерения проводились с помощью температурного эффекта Джоуля-Томсона для давлений в диапазоне от 1.5 до 4 атм. при 4 различных температурах газа в диапазоне от $\SI{10}{\celsius}$ до $\SI{70}{\celsius}$. Из графика зависимости коэффициента Джоуля-Томсона от обратной температуры газа на входе в трубку получены значения коэффициентов $a = (0.76 \pm 0.06) \frac{\text{Па}\cdot \text{м}^6}{\text{моль}^2\cdot K}, b = (0.32 \pm 0.05)\cdot 10^{-3} \frac{\text{м}^3}{\text{моль}}$ и температура инверсии $T_\text{инв} = (571 \pm 93)K$. Полученные значения на порядок отличаются от табличных, из этого следует, что модель газа Ван-дер-Ваальса для углекислого газа в диапазоне температур от $\SI{10}{\celsius}$ до $\SI{70}{\celsius}$ не обладает достаточной точностью для определения численных характеристик.