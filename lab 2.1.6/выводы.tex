\section{Выводы}

Определено, что график зависимости $\Delta T(\Delta P)$ является линейным и при увеличении давления газа на входе в трубку уменьшается разность температур $\Delta T$ газа на входе и выходе из трубы, то есть газ охлаждается до более низких температур.

Определены значения коэффициентов $a, b$ в уравнении Ван-дер-Ваальса и температура инверсии для углекислого газа
\[ a = (0.76 \pm 0.06) \frac{\text{Па}\cdot \text{м}^6}{\text{моль}^2\cdot K}\]
\[ b = (0.32 \pm 0.05)\cdot 10^{-3} \frac{\text{м}^3}{\text{моль}}\]
\[T_\text{инв} = (571 \pm 93)K\]

Вычисленные значения существенно отличаются от табличных. Из этого следует, что модель газа Ван-дер-Ваальса для углекислого газа в диапазоне температур от $\SI{10}{\celsius}$ до $\SI{70}{\celsius}$ подходит для качественного описания физических процессов, но не является достаточно точной для вычисления количественных характеристик системы.



