\section{Введение}
В современных условиях количество легкоизвлекаемых запасов нефти уменьшается, а спрос на энергию возрастает. Появляется необходимость использовать методы повышения нефтеотдачи посредством снижения вязкости нефти. Один из способов - закачка углекислого газа в нефтяной пласт. Важнейшим фактором, влияющим на процесс вытеснения нефти, является изменение температуры $CO_2$ при его расширении в пористом материале пласта. Это явление описывается эффектом Джоуля-Томсона.

Углекислый газ при температурах, типичных для нефтяных пластов, охлаждается при расширении. Это охлаждение может оказывать как положительное, так и отрицательное влияние на повышение нефтеотдачи. Снижение температуры способствует конденсации углеводородов из нефти, улучшая ее вытеснение. Но чрезмерное охлаждение приводит к образованию гидратов, увеличивающих вязкость нефти и ухудшающих вытеснение.

Температура инверсии углекислого газа позволяет точнее определить температуру газа для повышения эффективности нефтеотдачи. В настоящей работе использована модель реального газа Ван-дер-Ваальса для определения температуры инверсии по коэффициентам уравнения Ван-дер-Ваальса.