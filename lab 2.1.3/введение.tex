\section{Введение}

Вентиляция шахт горнодобывающей промышленности является критически важным аспектом обеспечения безопасности труда и эффективности производственных процессов. Поддержание приемлемого микроклимата, удаление взрывоопасных и токсичных газов, а также пыли – ключевые задачи, стоящие перед системами вентиляции шахт. В связи с большими глубинами, протяженностью горных выработок и, как следствие, значительными аэродинамическими сопротивлениями, вентиляция шахт оперирует с большими объемами воздуха, перемещаемыми под значительным давлением. В этих условиях адиабатические процессы, связанные с изменением температуры воздуха при его сжатии и расширении, оказывают существенное влияние на параметры вентиляционной сети.

В частности, адиабатическое нагревание воздуха при сжатии вентиляторами главного проветривания и последующее адиабатическое охлаждение при расширении в горных выработках приводит к изменению плотности воздуха, что, в свою очередь, влияет на распределение воздушных потоков, эффективность проветривания отдельных участков и общую энергетическую эффективность системы вентиляции. 

Целью настоящей работы является определение показателя адиабаты воздуха для расчета изменения температуры воздуха в шахтах при адиабатическом расширении для повышения энергетической эффективности системы вентиляции.