\section{Методика}

Известно из литературы \cite{Sivuhin}, что скорость звука в газе зависит от показателя адиабаты газа по следующему закону
\begin{equation}
    c = \sqrt{\gamma \frac{RT}{\mu}} \label{eq: c}
\end{equation}
где $\gamma$ - показатель адиабаты газа, $\gamma = \frac{c_p}{c_v}$, $c_p, c_v$ - молярные теплоемкости газа при постоянных давлении и объеме соответственно. $R = 8.31\frac{\text{Дж}}{\text{моль}\cdot K}$ - универсальная газовая постоянная. $T$ - температура газа, $\mu$ - молярная масса газа.

Из выражения \eqref{eq: c} следует, что показатель адиабаты вычисляется через выражение \eqref{eq: gamma}.
\begin{equation}
    \gamma = \frac{c^2 \mu}{RT} \label{eq: gamma}
\end{equation}

Показатель адиабаты зависит от количества степеней свободы идеального газа следующим образом

\begin{equation}
    \gamma = \frac{i+2}{i} \label{eq: i}
\end{equation}
Рассмотрим распространение звука в трубе с закрытыми противоположными концами. Звуковая волна испытывает многократные отражения от торцов. Звуковые колебания в трубе являются наложением всех отражённых волн. Резонанс наблюдается при условии, что на длине трубы помещается целое число длин полуволн:
\begin{equation}
    L = \frac{n\lambda}{2} \label{eq: L}
\end{equation}
где $L$ - длина трубы, $\lambda$ - длина волны, $n$ - натуральное число.

Скорость звука связана с его частотой следующим образом
\begin{equation}
    c = \lambda f \label{eq: c}
\end{equation}
$\lambda$ - длина волны, $f$ - частота волны, $c$ - скорость звука.

Чтобы определить скорость звука в воздухе, необходимо создать в трубе условия для возникновения резонанса. В работе представлены два способа реализации.

\textbf{1.} Создадим в трубе с подвижным концом звуковую волну постоянной частоты $f$, а значит и постоянной длины, что следует из выражения $\eqref{eq: c}$. Чтобы добиться выполнения условия \eqref{eq: L} (условие резонанса) необходимо менять длину трубы. Тогда по мере увеличения длины трубы будет наблюдаться последовательность резонансов, и для каждого резонанса выражение \eqref{eq: L} будет иметь вид \eqref{eq: L_nk}.
\[L_n = \frac{n\lambda}{2}\]
\begin{equation}
    L_{n+k} = \frac{n\lambda}{2} + \frac{k\lambda}{2} \label{eq: L_nk}
\end{equation}
$L_{n+k}$ - длина трубы на $k$ резонансе.

По угловому коэффициенту графика зависимости удлинения $L_{n+k} - L_n$ трубы от номера резонанса $k$ при частоте $f$ волны определяется длина звуковой волны, а затем по выражению \eqref{eq: c} вычисляется скорость звука в воздухе при частоте $f$. Построим данные графики для нескольких значений частоты волны, чтобы определить зависимость $c(f)$.

\textbf{2.} Используем трубу постоянной длины. В этом случае при изменении частоты $f$ звуковой волны изменяется и длина волны $\lambda$ согласно выражению \eqref{eq: c}. Значит, будет наблюдаться последовательность резонансов, и выражение \eqref{eq: L} примет вид
\begin{equation}
    L = \frac{\lambda_1 n}{2} = \frac{\lambda_2 (n+1)}{2} = \dots = \frac{\lambda_{k+1} (n+k)}{2} \label{eq: L2}
\end{equation}

Из \eqref{eq: c} и \eqref{eq: L2} имеем
\[f_1 = \frac{c}{\lambda_1} = \frac{c}{2L}n\]
\[f_2 = \frac{c}{\lambda_2} = \frac{c}{2L}(n+1) = f_1 + \frac{c}{2L}\]
\begin{equation}
    f_{k+1} = \frac{c}{\lambda_{k+1}} = \frac{c}{2L}(n+k) = f_1 + \frac{c}{2L}k \label{eq: fk}
\end{equation}

По угловому коэффициенту графика зависимости разности $f_{k+1} - f_1$ частоты текущего и первого резонансов от номера резонанса $k$ при некоторой температуре воздуха $T$ определяется скорость звука при температуре воздуха $T$. Построим данный график для нескольких значений температуры воздуха, чтобы определить, как зависит скорость звука от температуры газа.

Для некоторого значения скорости звука и соответствующей ему температуре газа вычислим показатель адиабаты для воздуха из выражения \eqref{eq: gamma}.