\section*{Аннотация}

Определены скорость звука в воздухе и показатель адиабаты воздуха. Измерения проводились двумя способами на двух установках и основывались на явлении резонанса, возникающем при отражении звуковой волны от торцов трубы. В процессе экспериментов температура воздуха изменялась в диапазоне от 20 до $60\celsius$, а частота звуковой волны в диапазоне от 2 до 6 $\text{КГц}$. По графикам зависимости разности частоты текущего и первого резонансов от номера резонанса для различных температур воздуха получены значения скоростей звука в диапазоне от 340 до $360\frac{\text{м}}{\text{с}}$. Затем по значениям скоростей звука для определенных температур вычислены показатели адиабаты для этих температур. Выяснено, что они не зависят от температуры воздуха в диапазоне от 20 до $60\celsius$, и среднее значение равно $(1370 \pm 5)\cdot 10^{-3}$.