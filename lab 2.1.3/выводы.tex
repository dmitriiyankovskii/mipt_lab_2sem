\section{Выводы}

Определено, что скорость звука в воздухе не зависит от частоты звуковой волны в диапазоне частот от $2$ до $6\text{КГц}$ и равна
\[ c = (346.8 \pm 1.8) \frac{\text{м}}{\text{с}}\]

Определено, что с ростом температуры воздуха скорость звука в нём увеличивается от 340 до $360\frac{\text{м}}{\text{с}}$. Это связано с увеличением кинетической энергии молекул и повышением интенсивности их взаимодействия.

Определено, что в диапазоне температур воздуха от 20 до $60\celsius$ показатель адиабаты остается постоянным и равным
\[\gamma = (1370 \pm 5)\cdot 10^{-3}\]

Определено, что воздух является двухатомным газом с 5 степенями свободы.